\documentclass[12pt]{article}
 \usepackage[hcentering,bindingoffset=20mm]{geometry}
 \usepackage{placeins}
 \usepackage[numbib]{tocbibind}
 \usepackage{rotating}
\usepackage[square,sort,comma,numbers]{natbib}
 \usepackage{graphicx}
 \usepackage{tabularx}
 \linespread{1.3}
 \usepackage{gensymb}
 \usepackage{longtable}
 \usepackage{lscape}
 \usepackage{url}
 \addtolength{\textwidth}{2cm}
 \addtolength{\hoffset}{-1cm}
 
 
 \addtolength{\textheight}{2cm}
 \addtolength{\voffset}{-1cm}
 \setlength{\parindent}{0pt}
 
\title{Gonyalacales phylogeny. $^{1}$}
\author{me}
\date{}

\begin{document}
\maketitle
\paragraph{}Anna Liza Kretzschmar$^{2}$\\
Climate Change Cluster (C3), University of Technology Sydney, Ultimo, 2007 NSW, Australia, anna.kretzschmar@uts.edu.au
\paragraph{}Aaron Darling \\
Ithree institute (i3), University of Technology Sydney, Ultimo, 2007 NSW, Australia
\paragraph{}Shauna Murray\\ 
Climate Change Cluster (C3), University of Technology Sydney, Ultimo, 2007 NSW, Australia
\newpage
\section{Abstract}
\newpage

\section{Introduction}
Dinoflagellates represent an ancient lineage of the eukaryotic branch of life. Elucidating the evolution of the dinoflagellates is key to answearing a number of questions. They play a role in several important ecological factors, as many members are harmful algal bloom organisms. Several members are prolific neurotoxin producers and have been implicated in seafood intoxication. The neurotoxins enter the food chain as the protists are consumed, commonly along with the substrate they inhabitate, such as macroalgae, which are then passed from one vector to another to higher trophic levels till they reach humans. These single celled algae are the causative agent for many seafood related illnesses, such as ciguatera fish poisoning (CFP), diarrhetic shellfish poisoning (DSP) and paralytic shell fish poisoning (PSP).
Several studies have attempted to do this with ribosomal DNA genes, mitochondrial genes and protein coding genes by concatenation to varying degrees of success. While the evolutionary relationship of most orders within the dinoflagellates can be estimated with reasonable certainty, one order has consistently escaped scrutiny - the gonyaulacales. Analyses for this order have consistently yielded long branch attraction, low confidence values and inconsistent taxon resolution. Yet illuminating the evolution of the gonyaulaceles is key, as neurotoxin production is prevalent in this order. 
\emph{Alexandrium} and xx spp. produce saxitoxin, a trialkyl tetrahydropurine in structure, which is responsible for PSP. Murray et al. (2012) have demonstrated that the gene complex responsible for saxitoxin production derives from horizontal gene transfer (HGT) from cyanobacteria.
A common type of toxin found within the dinoflagellates, ans especially the gonyaulacales, are polyketide toxins. Specifically the ciguatoxins produced by some \emph{Gambierdiscus} spp. are of interest as CFP is a neglected tropical disease with high associated morbidity, which is predicted to increase in prevalence with the advent of climate change.

discuss disagreement between authorities like ncbi and algaebase

incomplete lineage sorting 
\newpage

\section{Materials and methods}

\FloatBarrier
\begin{table}
\caption{Transcriptomes used for study along with taxonomic placement at family level and source. Family level placement derived from algaebase. MMETSP abbreviation for marine Microbial eukaryotic transcriptome sequencing project, by Moore Foundation.}
\label{tbl:Transcriptomes}
\begin{tabular}{  | p{3cm} |p{4cm} | p{2cm} | p{3cm} | p{2cm} | p{2cm} | p{2cm} |}
\hline
\textbf{Family}&\textbf{Species}&\textbf{Strain}&\textbf{ID}&\textbf{CEGMA score}&\textbf{BUSCO single complete}&\textbf{Reference}\\
\hline
 \multicolumn{7}{| c |}{Gonyaulacales transcriptomes}\\
    \hline
   Ceratiaceae&\emph{Ceratium fusus}&PA161109 &MMETSP1074&79.44&202&\citep{keeling2014marine}\\
        \hline
  Crypthecodiniaceae&\emph{Crypthecodinium cohnii}&Seligo&MMETSP0326\_2&83.87&202&\citep{keeling2014marine}\\
        \hline
    &\emph{Alexandrium catanella}&OF101&MMETSP0790&62.10&178&\citep{keeling2014marine}\\
        \hline
    &\emph{Alexandrium monilatum}&JR08&MMETSP0093&77.82&208&\citep{keeling2014marine}\\
        \hline
&\emph{Pyrodinium bahamense}&pbaha01&MMETSP0796&79.84&197&\citep{keeling2014marine}\\
        \hline
Gonyaulacaceae&\emph{Gambierdiscus australes}&CAWD149&MMETSP0766\_2&73.79&172&\citep{keeling2014marine}\\
        \hline
    &\emph{Gambierdiscus lapillus}&HG4&This study&85.08&227& \\
        \hline
    &\emph{Gambierdiscus} cf. \emph{silvae}&HG5&This study&77.42&215& \\
        \hline
    &\emph{Gonyaulax spinifera}&CCMP409&MMETSP1439&67.34&131&\citep{keeling2014marine}\\
        \hline
    &\emph{Lingulodinium polyedra}&CCMP1738&MMETSP1032&77.82&209&\citep{keeling2014marine}\\
        \hline
Protoceratiaceae&\emph{Protoceratium reticulatum}&CCCM535=CCMP1889&MMETSP0228&61.69&176&\citep{keeling2014marine}\\
    \hline
    \multicolumn{7}{| c |}{Peridiniales transcriptomes}\\
 \hline
 \multicolumn{7}{| c |}{Outgroup transcriptomes}\\
 \hline
Dinophysiaceae&\emph{Dinophysis acimunata}&DAEP01&MMETSP0797&63.71&178&\citep{keeling2014marine}\\
        \hline
Dinophyceae incertae sedis&\emph{Azadinium spinosum}&3D9&MMETSP1036\_2&80.65&193&\citep{keeling2014marine}\\
        \hline
Gymnodiniales&\emph{Karenia brevis}&CCMP2229&MMETSP0030&80.65&184&\citep{keeling2014marine}\\
    \hline
\end{tabular}
\end{table}
Azadinium Order Dinophyceae incerta sedis according to algaebase, gonyaulacales from NCBI taxonomy


\newpage
\section{Results}

%\FloatBarrier 
%\begin{figure} 
%\includegraphics[scale=.3]{HG7-env.png} 
%\caption{Detection of \emph{G. lapillus} per spatial replicate at each sampling site, cell numbers as normalised to HG7 standard curve (Fig. ~\ref{fig:lapiStd}B). Spatial replicates coloured as per macroalgal substrate,\ where \emph{Chnoospora} sp. are green, \emph{Sargassum} sp. are blue, \emph{Padina} sp. are red and mixed macroalgal substrates are yellow (see table ~\ref{tbl:MacroalgaeTable}).} 
%\label{fig:envHG7}
%\end{figure} 
%\FloatBarrier

\newpage
\section{Discussion}
Good for discussing findings which are different to literature to date Arroyave '11 Phylogenetic relationships and the temporal context for the diversification
of African characins of the family Alestidae (Ostariophysi: Characiformes): Evidence from DNA sequence data
- Zhang 07 use SSU cox1 and cob but only one or two gonya
- bachvaroff '14 74 protein matrix for phylo. they used concatenation approach and booted duplicate copies by selecting longest and some other rationales. C, chonii wasn't well resolved for them. Only moderate support for gonyaulacpods and prorocentrales but support dropped when gblocks trimming was put into effect. A.tamarenseshowed entirely differently placed copies of gene suggesting gene duplication confounding the analysis.  ref Bachvaroff and Place, 2008; Bachvaroff et al.,2009; Shoguchi et al., 2013 for gene duplication
-bachvaroff '11 concatenation approach 17 rDNA genes. only one gonya.
-derelle '16 stramenopile phylo but use concatenation which behaves badly with recombination/incomplete lineage sorting.




\newpage
\section{Conclusion}
\newpage

\section{Acknowledgements}
\bibliographystyle{acm}
\bibliography{gonya.bib}


\end{document}